\documentclass[12pt]{article}

\usepackage[utf8]{inputenc}
\usepackage{amsmath,amssymb,bbm}
\usepackage{tikz-cd}
\usepackage[margin=1in]{geometry}
\usepackage{lmodern}
\usepackage{microtype}
\usepackage{fancyhdr}
\usepackage{graphicx}
\usepackage[onehalfspacing]{setspace}
\usepackage{titlesec}
\usepackage{hyperref}
\usepackage{enumitem}

\pagestyle{fancy}
\fancyhf{}
\fancyhead[R]{\thepage}
\fancyhead[L]{}
\title{\includegraphics[scale=0.1]{logo.png}\\[0.5cm]\textbf{\Huge Final Exam}\\ \Large Linear Algebra\\ \Large July 23, 2023}
\author{\Large Departamento de Matemáticas, FIUBA}
\date{}

\titleformat{\section}{\large\bfseries}{}{0pt}{}
\titlespacing*{\section}{0pt}{\baselineskip}{0.5\baselineskip}

\begin{document}
	
	\maketitle
	\thispagestyle{empty}
	
	\section{Instructions}\label{sec:instructions}\sectionmark{Instructions}
	\begin{enumerate}
		\item All solutions must be presented with rigor and clarity. Justify your answers thoroughly.
		\item When referencing any results from the course, provide proper citations or references.
		\item Maintain a concise and legible writing style throughout your answers.
		\item Best of luck!
	\end{enumerate}
	
	\section{Exercise 1}\label{sec:e1}\sectionmark{Exercise 1}
	Let \(V\) be a \(K\)-vector space of dimension \(n\), and let \(C = \{v_1, \ldots, v_r\} \subset V\). Prove that:
	
	\begin{enumerate}[label=(\roman*)]
		\item There exists a basis \(B\) of \(V\) such that \(B \subseteq C\) if and only if \(C\) is a generating system of \(V\).
		\item There exists a basis \(B\) of \(V\) such that \(B \supseteq C\) if and only if \(C\) is linearly independent.
	\end{enumerate}
	
	\section{Exercise 2}\label{sec:e2}\sectionmark{Exercise 2}
	Let \(K\) be a field, and let \(B = \{v_1, \ldots, v_n\}\) be a basis of \(K^n\), where \(v_i = (v_{i,1}, \ldots, v_{i,n})\) for \(i = 1, \ldots, n\). Let \(\Delta = \det((v_{i,j})_{1\leq i,j \leq n})\). For \(j = 1, \ldots, n\), the map \(\varphi_j : K^n \rightarrow K\) is defined as follows:
	
	\[
	\varphi_j (x_1, \ldots, x_n) = \Delta^{-1} \cdot \det
	\begin{pmatrix}
		v_{1,1} & v_{1,2} & \cdots & v_{1,n} \\
		\vdots & \vdots & \ddots & \vdots \\
		v_{j-1,1} & v_{j-1,2} & \cdots & v_{j-1,n} \\
		x_1 & x_2 & \cdots & x_n \\
		v_{j+1,1} & v_{j+1,2} & \cdots & v_{j+1,n} \\
		\vdots & \vdots & \ddots & \vdots \\
		v_{n,1} & v_{n,2} & \cdots & v_{n,n} \\
	\end{pmatrix}
	\]
	
	i) Verify that \(\varphi_j \in (K^n)^*\) for all \(1 \leq j \leq n\).
	
	ii) Prove that \(B^* = \{\varphi_1, \ldots, \varphi_n\} \subset (K^n)^*\) is the dual basis of \(B\).
	
iii) Let \(S = \langle v_1 - v_n, \ldots, v_{n-1} - v_n \rangle\). Find a basis for \(S^{o}\).

	
\end{document}
